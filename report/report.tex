\documentclass[a4paper, 11pt]{article}
\usepackage{graphicx}
\usepackage{amsmath}
\usepackage[pdftex]{hyperref}

% Lengths and indenting
\setlength{\textwidth}{16.5cm}
\setlength{\marginparwidth}{1.5cm}
\setlength{\parindent}{0cm}
\setlength{\parskip}{0.15cm}
\setlength{\textheight}{22cm}
\setlength{\oddsidemargin}{0cm}
\setlength{\evensidemargin}{\oddsidemargin}
\setlength{\topmargin}{0cm}
\setlength{\headheight}{0cm}
\setlength{\headsep}{0cm}

\renewcommand{\familydefault}{\sfdefault}

\title{Machine Learning 2014: Project 2 - Classification Report}
\author{lukasbi@student.ethz.ch\\ ajenal@student.ethz.ch\\ harhans@student.ethz.ch\\}
\date{\today}

\begin{document}
\maketitle

\section*{Experimental Protocol}
%Suppose that someone wants to reproduce your results. Briefly describe the steps used to obtain the
%predictions starting from the raw data set downloaded from the project website. Use the following
%sections to explain your methodology. Feel free to add graphs or screenshots if you think it's
%necessary. The report should contain a maximum of 2 pages.

\section{Tools}
For this exercise we used mainly Matlab. All the statistic computation was performed with Matlab. 

\section{Algorithm}
Describe the algorithm you used for classification.
% matlab integrated fitcsvm  

\section{Features}
Did you perform any preprocessing on the features? What feature transforms did you use?

% rbf kernel

\section{Parameters}
How did you find the parameters of your model? (What parameters have you searched over, cross validation procedure, $\ldots$)

% outlier fraction, boxconstraints, kernelscale

\section{Lessons Learned} What other algorithms did you try out that didn't work well?
Why do you think they performed worse than what you used for your final submission?

% e.g. linear and quadratic kernel ! %

\end{document}
